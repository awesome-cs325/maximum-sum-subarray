\documentclass[letWterpaper,10pt,titlepage]{article}

\usepackage{graphicx}
\usepackage{amssymb}
\usepackage{amsmath}
\usepackage{amsthm}
\usepackage{alltt}
\usepackage{float}
\usepackage{color}
\usepackage{url}
\usepackage{balance}
\usepackage[TABBOTCAP, tight]{subfigure}
\usepackage{enumitem}
\usepackage{pstricks, pst-node}
\usepackage{geometry}
\geometry{textheight=10in, textwidth=7.5in}
\newcommand{\cred}[1]{{\color{red}#1}}
\newcommand{\cblue}[1]{{\color{blue}#1}}
\usepackage{hyperref}
\usepackage{textcomp}
\usepackage{listings}

%% The following metadata will show up in the PDF properties
\hypersetup{
  colorlinks = true,
  urlcolor = black,
  pdfkeywords = {cs325 ``Analysis of Algorithms'' subarray sum},
  pdftitle = {CS325 Project 1},
  pdfsubject = {CS325 Project 1},
  pdfpagemode = UseNone
}

\definecolor{comment}{rgb}{0, 0.6, 0}

\lstset{
    belowcaptionskip=1\baselineskip,
    breaklines=true,
    frame=single,
    xleftmargin=\parindent,
    language=Python,
    showstringspaces=false,
    basicstyle=\footnotesize\ttfamily,
    identifierstyle=\color{blue},
    stringstyle=\color{orange},
    commentstyle=\color{comment},
}

\parindent = 0.0 in
\parskip = 0.2 in


\begin{document}

\title{CS 325 Project 1: Maximum Sum Subarray}
\author{Project Group 25:\\Jonathon Moore\\Jaden Diefenbaugh\\Kenneth Hafdahl}
\maketitle

\section{Theoretical Run-time Analysis}

\subsection{Enumeration}

\begin{lstlisting}
def maxSumSubarray(array):
    bestSum = array[0]
    #lower index of max sum array
    bestLower = 0
    #upper index of max sum array
    bestUpper = 0
    for n in range(len(array)):
        for m in range(n, len(array)):
            currentSum = 0
            for p in range(n, m+1):
                currentSum = currentSum + array[p]
            if currentSum > bestSum:
               bestSum = currentSum
               bestUpper = m
               bestLower = n
\end{lstlisting}

The first two loops of the enumeration algorithm determine the starting and ending indices of the subarray to be summed. There is one subarray of length $n$ generated, two arrays of length $n-1$, three arrays of length $n-2$, and so on. This totals $1, 2, 3, ..., n={n(n+1)}/2$ arrays which is $\Theta(n^2)$ complexity. The innermost loop sums the values between the two indices to determine if the current array is the maximum sum subarray. This is an $\Theta(n)$ operation that is performed for each of the subarrays, making this algorithm $\Theta(n^3)$.

\subsection{Better Enumeration}

\begin{lstlisting}
def maxSumSubarray(array):
    bestSum = array[0]
    #lower index of max sum array
    bestLower = 0
    #upper index of max sum array
    bestUpper = 0
    currentSum = 0
    for n in range(len(array)):
        lastSum = 0
        for m in range(n, len(array)):
            if m == n:
               currentSum = array[n]
            else:
                currentSum = lastSum + array[m]
            lastSum = currentSum
            if currentSum > bestSum:
               bestSum = currentSum
               bestUpper = m
               bestLower = n
\end{lstlisting}

The next version of the enumeration method removes the innermost addition loop. This is achieved by keeping a running sum for each subarray. The sum is resent when the lower bound is increased and the size of the array is zero. This optimization allows the algorithm to achieve a $\Theta(n^2)$ complexity.

\subsection{Divide and Conquer}

\begin{lstlisting}
# the combining half of algorithm3
def mergeHalves(lowerHalf, upperHalf):
    #set bounds of new combined array
    j = lowerHalf[0]
    k = upperHalf[1]

    #compute max sum of combined arrays
    # it will either be:
    #   the max sum of lowerHalf
    #   the max sum of upperHalf
    #   the combined sums upper end of lowerHalf and lower end of upperHalf
    if lowerHalf[2] > upperHalf[2]:
        newMaxSum = lowerHalf[2]
        maxSumJ = lowerHalf[3]
        maxSumK = lowerHalf[4]
    else:
        newMaxSum = upperHalf[2]
        maxSumJ = upperHalf[3]
        maxSumK = upperHalf[4]
    suffixMaxSum = lowerHalf[7] + upperHalf[5]
    if suffixMaxSum > newMaxSum:
        newMaxSum = suffixMaxSum
        maxSumJ = lowerHalf[8]
        maxSumK = upperHalf[6]

    return [j, k, newMaxSum, maxSumJ, maxSumK, -1, -1, -1, -1]


# recursive solution to finding the max sum (algorithm 3)
def algorithm3(array, lower, upper):
    upperSum = [];
    lowerSum = [];
    # format of array that holds return data:
    # Bounds of subarray:
    # [0] lower bound of subarray
    # [1] upper bound of subarray
    # MaxSum data within that array:
    # [2] bestMaxSum
    # [3] lower bound of range that holds bestMaxSum
    # [4] upper bound of range that holds bestMaxSum
    # Best sum that overlaps along the lower bound of subarray:
    # [5] bestMaxSum that includes lower bound of the subarray
    # [6] upper bound of the range that creates bestMaxSum(Lower)
    # Best sum that overlaps along the upper bound of subarray:
    # [7] bestMaxSum that includes upper bound of the subarray
    # [8] lower bound of the range that creates bestMaxSum(Upper)
    bestMaxSum = [lower, upper, -99999, 0, 0, -99999, 0, -99999, 0]
    if lower == upper:  #size of subarray is 1
        bestMaxSum = [lower, upper, array[lower], lower, upper, array[lower], lower, array[upper], upper]
    else:   # split array into 2 subarrays and recursively merge their answers
        mid = math.floor((upper + lower) / 2)
        lowerSum = algorithm3(array, lower, mid)
        upperSum = algorithm3(array, mid+1, upper)
        bestMaxSum = mergeHalves(lowerSum, upperSum)


        #find best sum that covers the lower suffix of the joined array
        best_here = 0
        best_so_far = 0
        best_pos = 0

        for i in range(bestMaxSum[0], bestMaxSum[1]+1):
            best_here = array[i] + best_here
            if best_here > best_so_far:
                best_so_far = best_here
                best_pos = i

        bestMaxSum[5] = best_so_far
        bestMaxSum[6] = best_pos

        #find best sum that covers the upper suffix of the joined array
        best_here = 0
        best_so_far = 0
        best_pos = 0
        for j in range(bestMaxSum[1], bestMaxSum[0]-1, -1):
            best_here = array[j] + best_here
            if best_here > best_so_far:
                best_so_far = best_here
                best_pos = j

        bestMaxSum[7] = best_so_far
        bestMaxSum[8] = best_pos

    return bestMaxSum
\end{lstlisting}

The divide and conquer approach is a recursive algorithm that has a base case in which the size of the subarray is 1. If the subarray size is greater than 1, a midpoint is calculated for the divide and the algorithm is recursively called on either half. A helper function is then called that logically merges the two halves. The each subarray has a maximum sum sequence and two maximum sum sequences that start or end on the subarray bounds. If the combination of two subarrays yeilds a summation that is larger than either if the individual summations, that merge is returned. The division portion of the algorithm is $O(log_2(n))$ and the merging portion is $O(1)$.

\subsection{Linear-Time}

\begin{lstlisting}
\end{lstlisting}


\section{Proof of Correctness}

\section{Testing}
testing

\section{Experimental Analysis}
experimental

\section{Extrapolation and Interpretation}
extrapolation
\begin{figure}
    \centering
    \includegraphics[width=0.75\textwidth]{../bin/testrun.eps}
\end{figure}
\end{document}
